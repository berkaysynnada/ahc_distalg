\documentclass[11pt]{beamer}  % only frames
\usetheme{Warsaw}
\usepackage{graphicx}
\usepackage{hyperref}
\usepackage[utf8]{inputenc}
\usepackage[english]{babel}
\usepackage{amsmath}
\usepackage{amsfonts}
\usepackage{amssymb}
\usepackage{algorithm2e}

\title[Distributed Algorithm on AHCv2]{Distributed Algorithm on AHCv2: Waves: Tarry's Traversal and Tree, Release V1.0.0}
\author[Berkay Şahin]{Berkay Şahin\\}
\date{\today}

\begin{document}

\begin{frame}[plain]
    \titlepage
\end{frame}

\begin{frame}{Outline}
    \tableofcontents
\end{frame}

\section{Introduction}
\begin{frame}{Introduction}
    \textbf{Context:} Introduction to the challenges and importance of efficient traversal algorithms in distributed systems.
    \begin{itemize}
        \item Importance of Tarry's Algorithm and Tree Algorithm in network exploration and rapid network setup.
        \item Discussion on their theoretical implications and expected behaviors in different network topologies.
    \end{itemize}
    \textbf{Problem Statement:}
    \begin{itemize}
        \item In distributed systems, effective network traversal and spanning tree construction are crucial.
        \item The primary challenge is to design protocols that minimize communication overhead and ensure completion.
    \end{itemize}
    \textbf{Significance and Necessity:}
    \begin{itemize}
        \item These algorithms provide mechanisms for efficient data dissemination and network reconfiguration.
        \item Solving this problem enables efficient resource management and fault diagnosis.
    \end{itemize}
\end{frame}

\section{Implementation and Methodology}
\begin{frame}{Implementation and Methodology}
    \textbf{Simulation Setup:}
    \begin{itemize}
        \item Description of the AHCv2 simulation environment.
        \item Network topologies used: linear, tree, and random graphs.
        \item Metrics captured: message count, traversal time, overhead.
    \end{itemize}
    \textbf{Methodological Approach:}
    \begin{itemize}
        \item Nodes initialized with either Tarry's or the Tree Algorithm.
        \item Simulations to measure efficiency and coverage across network topologies.
    \end{itemize}
\end{frame}

\section{Results}
\begin{frame}{Results}
    \textbf{Theoretical Results:}
    \begin{itemize}
        \item Tarry's Algorithm is expected to use about 2E messages for complete traversal.
        \item Tree Algorithm is designed for quick spanning tree formation but may incur higher message overhead.
    \end{itemize}
    \textbf{Expected Outcomes:}
    \begin{itemize}
        \item Comprehensive coverage by Tarry's Algorithm.
        \item Rapid spanning tree formation by the Tree Algorithm.
    \end{itemize}
\end{frame}

\section{Discussion}
\begin{frame}{Discussion}
    Discuss the implications of the findings:
    \begin{itemize}
        \item Suitability of Tarry’s Algorithm for comprehensive network exploration.
        \item Tree Algorithm's application in scenarios requiring rapid deployment.
    \end{itemize}
    \textbf{Practical Recommendations:}
    \begin{itemize}
        \item Tarry's Algorithm for detailed network exploration.
        \item Tree Algorithm for rapid setup in structured networks.
    \end{itemize}
\end{frame}

\section{Conclusion}
\begin{frame}{Conclusion}
    Summarize the theoretical insights and their practical implications:
    \begin{itemize}
        \item Both algorithms serve critical but distinct roles in distributed systems.
        \item Future research directions, such as reducing overhead and adapting to dynamic network conditions.
    \end{itemize}
    \textbf{Final Thoughts:}
    \begin{itemize}
        \item Theoretical exploration highlights the need for empirical validation.
        \item Importance of ongoing research in efficient traversal algorithms.
    \end{itemize}
\end{frame}

\begin{frame}{References}
    % Add your references here
\end{frame}

\begin{frame}{Questions}
    \centering \Large
    Thank you!\\
    Questions?
\end{frame}

\end{document}
