\documentclass[11pt]{beamer}
\usetheme{Warsaw}
\usepackage{graphicx}
\usepackage{hyperref}
\usepackage[utf8]{inputenc}
\usepackage[english]{babel}
\usepackage{amsmath}
\usepackage{amsfonts}
\usepackage{amssymb}
\usepackage{algorithm2e}

\title[Distributed Algorithm on AHCv2]{Distributed Algorithm on AHCv2: Waves: Tarry's Traversal and Tree Algorithms, Release V1.0.0}
\author[Berkay Şahin]{Berkay Şahin}
\date{\today}

\begin{document}

\begin{frame}[plain]
    \titlepage
\end{frame}

\begin{frame}{Outline}
    \tableofcontents
\end{frame}

\section{Introduction}
\begin{frame}{Introduction}
    \textbf{Context:}
    \begin{itemize}
        \item Introduction to the challenges and importance of efficient traversal algorithms in distributed systems.
        \item Essential for managing complex networks involved in distributed systems.
    \end{itemize}
\end{frame}

\begin{frame}{Problem Statement}
    \textbf{Problem Statement:}
    \begin{itemize}
        \item The need for effective network traversal and spanning tree construction.
        \item Importance of designing protocols that minimize communication overhead.
        \item Ensuring completion even in dynamic network topologies.
    \end{itemize}
\end{frame}

\begin{frame}{Significance and Necessity}
    \textbf{Significance and Necessity:}
    \begin{itemize}
        \item Why these algorithms are crucial for system robustness and performance.
        \item Impact on efficient resource management and fault diagnosis.
    \end{itemize}
\end{frame}

\section{Tarry's Algorithm}
\begin{frame}{Overview of Tarry's Algorithm}
    \textbf{Tarry's Traversal Algorithm:}
    \begin{itemize}
        \item Designed for undirected graphs.
        \item Ensures every node is visited exactly once.
    \end{itemize}
\end{frame}

\begin{frame}{Principles and Mechanism}
    \textbf{Principles and Mechanism:}
    \begin{itemize}
        \item Simple token-passing mechanism.
        \item Each channel is visited twice to ensure completeness.
    \end{itemize}
\end{frame}

\begin{frame}{Pseudocode of Tarry's Algorithm}
    \textbf{Pseudocode:}
    \begin{algorithm}[H]
    \SetAlgoLined
    \If{node is initiator}{
     send token to an arbitrary neighbor\;
    }
    \While{token is received from neighbor}{
     mark self as visited\;
     \eIf{there are unvisited neighbors}{
      choose an unvisited neighbor and send the token\;
     }{
      return token to sender\;
     }
    }
    \caption{Tarry's Algorithm}
    \end{algorithm}
\end{frame}

\begin{frame}{Benefits and Evaluation}
    \textbf{Benefits and Evaluation:}
    \begin{itemize}
        \item Ensures complete network coverage.
        \item Minimizes message overhead by avoiding revisits.
    \end{itemize}
\end{frame}

\section{Tree Algorithm}
\begin{frame}{Overview of Tree Algorithm}
    \textbf{Tree Algorithm:}
    \begin{itemize}
        \item Designed for acyclic networks.
        \item Focuses on quick spanning tree formation.
    \end{itemize}
\end{frame}

\begin{frame}{Principles and Mechanism}
    \textbf{Principles and Mechanism:}
    \begin{itemize}
        \item Waits for messages from all neighbors except one.
        \item Decision making by exactly two nodes in the network.
    \end{itemize}
\end{frame}

\begin{frame}{Pseudocode of Tree Algorithm}
    \textbf{Pseudocode:}
    \begin{algorithm}[H]
    \SetAlgoLined
    \ForEach{node in network}{
     Listen for messages from all neighbors except one\;
     \If{messages received from all but one neighbor}{
      Select the neighbor with no message as parent\;
      Send message to selected parent\;
     }
     \If{message received from parent}{
      Finalize decision\;
     }
    }
    \caption{Tree Algorithm}
    \end{algorithm}
\end{frame}

\begin{frame}{Benefits and Evaluation}
    \textbf{Benefits and Evaluation:}
    \begin{itemize}
        \item Efficient message utilization.
        \item Quick decision-making process.
    \end{itemize}
\end{frame}

\section{Implementation and Methodology}
\begin{frame}{Simulation Setup}
    \textbf{Simulation Setup:}
    \begin{itemize}
        \item Description of the AHCv2 simulation environment.
        \item Details on network topologies used: linear, tree, and random graphs.
    \end{itemize}
\end{frame}

\begin{frame}{Methodological Approach}
    \textbf{Methodological Approach:}
    \begin{itemize}
        \item Nodes initialized with specific algorithms.
        \item Monitoring and capturing efficiency and coverage metrics.
    \end{itemize}
\end{frame}

\section{Results}
\begin{frame}{Theoretical Results}
    \textbf{Theoretical Results:}
    \begin{itemize}
        \item Expected message usage and network traversal completeness for Tarry's Algorithm.
        \item Speed and message overhead implications for the Tree Algorithm.
    \end{itemize}
\end{frame}

\section{Discussion}
\begin{frame}{Implications of Findings}
    \textbf{Implications of Findings:}
    \begin{itemize}
        \item Suitability of Tarry's Algorithm for detailed network exploration.
        \item Applicability of the Tree Algorithm for rapid deployment in structured networks.
    \end{itemize}
\end{frame}

\begin{frame}{Practical Recommendations}
    \textbf{Practical Recommendations:}
    \begin{itemize}
        \item Application scenarios for each algorithm based on their strengths.
    \end{itemize}
\end{frame}

\section{Conclusion}
\begin{frame}{Conclusion}
    \textbf{Summary of Findings:}
    \begin{itemize}
        \item Comparative analysis of Tarry's and the Tree Algorithms.
        \item Future research directions for reducing overhead and improving adaptability.
    \end{itemize}
\end{frame}

\begin{frame}{Final Thoughts}
    \textbf{Final Thoughts:}
    \begin{itemize}
        \item Necessity of empirical validation of theoretical predictions.
        \item Importance of continued research in distributed systems traversal algorithms.
    \end{itemize}
\end{frame}

\begin{frame}{References}
    % Add your references here
\end{frame}

\begin{frame}{Questions}
    \centering \Large
    Thank you! Any questions?
\end{frame}

\end{document}
